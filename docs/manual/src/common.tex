\begin{titlepage}
  \begin{center}

  {\Huge Open Game Module}

  \vspace{25mm}

  \includegraphics[width=0.90\textwidth,height=\textheight,keepaspectratio]{img/SPARKLETRON.png}

  \vspace{25mm}

  \today

  \vspace{15mm}

  {\Large Jay Convertino}

  \end{center}
\end{titlepage}

\tableofcontents

\newpage

\section{Introduction}

\par
Open Game Module is a open source expansion card for the Colecovision. This module is Super Game Module (SGM) compatable.
This means all SGM games will run when this module is used with a Colecovision. Includes the same 32KB of RAM. The sound
is expanded with a Yamaha YMZ284 instead of the original AY chip.

\subsection{Fix for opcode games}

Opcode super game module mega cart games have a particular check for the sound chip in there startup routine.

\begin{enumerate}
  \item Set AY register address to 0x00
  \item Write the value 0xAA to AY (register 0x00).
  \item Set AY register address to 0x02
  \item Write the value 0x55 to AY (register 0x02).
  \item Set AY register address to 0x00
  \item Read the value from AY (register 0x00).
  \item Compare to value originally written (0xAA), fail if not matching
  \item Set AY register address to 0x02
  \item Read the value from AY (register 0x02).
  \item Compare to value originally written (0x55), fail if not matching.
\end{enumerate}

The hack fix for my setup is to only have 4 8 bit regiters for 0,1,2,3,4,5,6, and 7. The mapping is
AY Software Register ADDRESS => Cache Register ADDRESS

\begin{itemize}
  \item 0 => 0
  \item 1 => 0
  \item 2 => 1
  \item 3 => 1
  \item 4 => 2
  \item 5 => 2
  \item 6 => 3
  \item 7 => 3
\end{itemize}

\subsection{Specifications}

\par
\begin{itemize}
  \item 32 KB of RAM
  \item YMZ284 Sound Chip
  \item MAX7000S CPLD (EPM7064SLC)
  \item PCB, two layer
\end{itemize}

\subsection{Parts List}

\subsubsection{electronics}
\begin{footnotesize}
\begin{longtable}{ |*{4}{c|} }
\hline
{Item} & {Qty} & {Reference(s)} & {Value} \\
\hline
{1} & {1} & {J1} & {Connector, 60 pin} \\
\hline
{2} & {4} & {C1, C2, C3, C4} & {100pF} \\
\hline
{3} & {1} & {U1} & {EPM7064SLC-10} \\
\hline
{4} & {1} & {U2} & {YMZ284} \\
\hline
{5} & {1} & {U3} & {HM62256BLP} \\
\hline
{6} & {1} & {R5} & {470R} \\
\hline
{7} & {1} & {R6} & {4k7} \\
\hline
{8} & {4} & {R1, R2, R3, R4} & {1k} \\
\hline
{9} & {1} & {J2} & {Pin Header, 10 pin} \\
\hline
\end{longtable}
\end{footnotesize}

\subsubsection{hardware}
\begin{footnotesize}
\begin{longtable}{ |*{4}{c|} }
\hline
{Item} & {Qty} & {Reference(s)} & {Value} \\
\hline
{1} & {4} & {S1} & {\#2-32 x 1/4 Fastenal 0148209, screw} \\
\hline
{2} & {1} & {TOP} & {top.stl 100.00} \\
\hline
{3} & {1} & {BOTTOM} & {bottom.stl 100.00} \\
\hline
\end{longtable}
\end{footnotesize}

\section{Building}

\par
This document assumes some Electrical Engineering knowledge. Building circuits is not
trivial due to the mix of SMD and through hole components. What follow are general
steps to build the Mini Colecovision

\begin{itemize}
  \item Create PCB from schematic/gerber/open\_game\_module.zip
  \item Populate PCB
  \item Power up and program CPLD
  \item Build your own case
\end{itemize}

\subsection{Dependencies}

\par
The following are the dependencies needed to build the firmware and PCB for the system.

\begin{itemize}
  \item Quartus 13.0 sp1
  \item python 3.X
  \item KiCAD v8.X
\end{itemize}

\subsubsection{Open Game Module Glue File List}
\begin{itemize}
\item src
	\begin{itemize}
	\item {'src/open\_game\_module.v': {'file\_type': 'verilogSource'}}
	\end{itemize}
\item constr
	\begin{itemize}
	\item {'constr/open\_game\_module.sdc': {'file\_type': 'SDC'}}
	\end{itemize}
\item tb
	\begin{itemize}
	\item {'tb/tb\_open\_game\_module.v': {'file\_type': 'verilogSource'}}
	\end{itemize}
\end{itemize}


\subsubsection{Fusesoc}
\par
Fusesoc is used for the simulation target only. There are no build targets due to the use of Quartus 13.0sp1.
This makes the use of it a bit silly. It does make it easier to use in future projects where the RAM,ROM,CPU,VDP,
and Sound chips are also IP cores.

\input{src/fusesoc/targets_fusesoc.tex}

\subsubsection{Quartus}
\par
This project uses the last version of Quartus that supports the MAX7000S series. The version is 13.0sp1.
The project is located at src/quartus13sp01/. Once you have the project open please follow the softwares steps
for building and programming the CPLD bitfile.

\subsection{PCB}

\par
The top has all the components of the circuit.

\begin{figure}[h!]
\caption{Top}
\centering
\includegraphics[width=0.90\textwidth,keepaspectratio]{img/ogm_top.png}
\end{figure}

\par
The bottom has no components.

\begin{figure}[h!]
\caption{Bottom}
\centering
\includegraphics[width=0.90\textwidth,keepaspectratio]{img/ogm_bottom.png}
\end{figure}

\subsection{3D Printed Case}

\par
The 3D printed case has been tested on two different printers. It has only been tested with ABS filament.
The parts list for the 3D printed case has the STL file name in the value and the XXX.X value is the scale size.
In general I recommend the following steps for assembly.

\begin{enumerate}
  \item Bottom, sit pcb aligned to screw holes.
  \item Top, sit top half on top and install screws.
\end{enumerate}

\subsection{Programming}

\par
There is one device that need to be programmed which is the CPLD (complex programmable logic device).
This is programmed with quartus using the JTAG header to upload the bitfile.

\subsubsection{CPLD}

\par
Quartus 13.0sp1 is the easiest way to build and program the MAX7000 CPLD. You will need an altera blaster.
I recommend the chinese clone blasters, they actually worked the best. While the worst was the Terasic blaster
which did not work at all. As for instructions on how to program it in Quartus, please see the software for details.

\newpage

\section{Usage}

\subsection{Directory Guide}

\par
Below highlights important folders from the root of the open\_game\_module.

\begin{enumerate}
  \item \textbf{docs} Contains all documentation related to this project.
    \begin{itemize}
      \item \textbf{datasheets} Contains all datasheets for components.
      \item \textbf{manual} Contains user manual and github page that are generated from the latex sources.
    \end{itemize}
  \item \textbf{img} Contains images of the project
  \item \textbf{schematic} KiCAD v8.X schematic and PCB designs
    \begin{itemize}
      \item \textbf{gerber} Contains gerber files and archives for production.
      \item \textbf{pdf} PDF schematic
    \end{itemize}
  \item \textbf{src} CPLD firmware source
    \begin{itemize}
      \item \textbf{open\_game\_module} Contains verilog source code and constraits
      \item \textbf{quartus13sp01} Quartus project used to generate firmware.
    \end{itemize}
\end{enumerate}
